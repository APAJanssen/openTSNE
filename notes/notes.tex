\documentclass[11pt]{article}

\usepackage{parskip}
\usepackage{geometry}
\usepackage[utf8]{inputenc}
\usepackage{amsmath, amssymb}
\usepackage{subcaption}
\usepackage{hyperref}

\geometry{
  a4paper,
  bottom=2.5cm,
  right =2.5cm,
  left  =2.5cm,
  top   =2.5cm,
}


\begin{document}
\title{t-SNE Notes}
\author{Pavlin Poličar}
\date{}
\maketitle

\section{Fast KL Divergence}

During computation of negative gradients, we do not know the value of the normalization term $Z$ during intermediate steps. Therefore, in order to compute the KL divergence of the embedding, we would need to iterate over all the $p_{ij}$s again, at the end of the gradient computation step. By rewriting the KL divergence in terms of unnormalized $q_{ij}$s, we can save on computation time.

\begin{align*}
KL(P \mid \mid Q) &= \sum_{ij} p_{ij} \log \frac{p_{ij}}{q_{ij}} \\
&= \sum_{ij} p_{ij} \log \left ( p_{ij} \frac{Z}{\hat{q}_{ij}} \right )
\intertext{where $\hat{q}_{ij}$ denotes the unnormalized values $q_{ij}$}
&= \sum_{ij} p_{ij} \log \frac{p_{ij}}{\hat{q}_{ij}} + \sum_{ij} p_{ij} \log Z
\end{align*}

\section{KL Divergence with exaggeration}

The implemented optimization methods don't have a notion of exaggeration, they simply take an affinity matrix $P$ containing the probabilities of points $j$ appearing close to $i$. Exaggeration is used to scale $P$ by some constant factor $\alpha$ to help separate clusters in the beginning of the optimization. These methods also compute the KL divergence, and as such, it is incorrect because we don't account for $\alpha$.

This section derives a quick and simple correction for the KL divergence error term so we can get the true error of the embedding even when $P$ is exaggerated.

\begin{align}
KL(P \mid \mid Q) &= \sum_{ij} p_{ij} \log \frac{p_{ij}}{q_{ij}}
\intertext{We need to introduce the scaling i.e. exaggeration factor $\alpha$}
&= \sum_{ij} \frac{\alpha}{\alpha}p_{ij} \log \frac{\alpha p_{ij}}{\alpha q_{ij}} \\
\intertext{Exaggeration means that the $p_{ij}$ terms get multiplied by $\alpha$, so we need to find an expression for the KL divergence that includes only $\alpha p_{ij}$ and $q_{ij}$ and some other factor that will correct for $\alpha$.}
&= \frac{1}{\alpha} \sum_{ij} \alpha p_{ij} \left ( \log \frac{\alpha p_{ij}}{q_{ij}} - \log \alpha \right ) \\
&= \frac{1}{\alpha} \left ( \sum_{ij} \alpha p_{ij} \log \frac{\alpha p_{ij}}{q_{ij}} - \sum_{ij} \alpha p_{ij}  \log \alpha \right )
\end{align}

The first term is computed by the negative gradient method (since it only knows about the scaled $P$), the second term can easily be computed post-optimization, allowing us to get the correct KL divergence.

\end{document}